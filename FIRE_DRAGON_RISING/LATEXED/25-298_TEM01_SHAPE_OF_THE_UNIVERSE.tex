\documentclass[11pt, letterpaper]{article}

% --- PACKAGES ---
\usepackage[utf8]{inputenc}
\usepackage[T1]{fontenc}
\usepackage{amsmath, amssymb, amsthm}
\usepackage{geometry}
\usepackage{graphicx}
\usepackage[hyphens]{url}
\usepackage[colorlinks=true, linkcolor=blue, citecolor=blue, urlcolor=blue]{hyperref}

% --- GEOMETRY ---
\geometry{letterpaper, margin=1in}

% --- MATH SYMBOLS ---
\newcommand{\R}{\mathbb{R}} % Real numbers
\newcommand{\Z}{\mathbb{Z}} % Integers
\newcommand{\RP}{\mathbb{RP}} % Real Projective Space
\newcommand{\PP}{\mathbb{P}} % Projective Space
\newcommand{\VCR}{\mathrm{VCR}} % Volumetric Cross-Ratio

% --- THEOREM ENVIRONMENTS ---
\newtheorem{theorem}{Theorem}[section]
\newtheorem{definition}[theorem]{Definition}
\newtheorem{corollary}[theorem]{Corollary}
\newtheorem{lemma}[theorem]{Lemma}
\newtheorem{proposition}[theorem]{Proposition}
\newtheorem{axiom}[theorem]{Axiom}

% --- TITLE & AUTHOR ---
\title{\textbf{Projective Reality: A Multiply-Connected Framework for Cosmological Geometry}}
\author{
    Oscar Wojciechowski-Prill \\
    \textit{Affiliation: Humanity} \\
    \href{mailto:everythingexpands@gmail.com}{everythingexpands@gmail.com}
}
\date{\today}

% --- BEGIN DOCUMENT ---
\begin{document}

\maketitle

\begin{abstract}
Modern cosmology faces persistent paradoxes—the Hubble tension, conflicting local-flatness (BAO) versus global-curvature (CMB) results, and anomalous large-angle correlations. We argue these are not measurement errors but inevitable symptoms of an unexamined assumption: that space is simply-connected.

We introduce \textbf{Projective Reality (PR)}, a complete geometric framework that resolves these paradoxes. PR is uniquely determined by observational incidence—the fundamental relations between points, lines, and planes as we perceive them. We establish two defining properties of PR through rigorous proof:
\begin{enumerate}
    \item \textbf{Topology:} The manifold of PR is necessarily \textbf{Real Projective 3-Space ($\RP^3$)}, a compact, multiply-connected topology systematically excluded from standard cosmology.
    \item \textbf{Invariant:} The fundamental invariant of PR is the \textbf{Volumetric Cross-Ratio (VCR)}, computable from tetrahedral volumes and preserved under all projective transformations.
\end{enumerate}
This $\RP^3$ topology inherently reconciles local flatness with global closure through antipodal identification. We derive five binary, falsifiable predictions—including antipodal matched circles in the CMB and transient ghost imaging—that follow necessarily from the topology alone, without free parameters.
Projective Reality is not a model among alternatives but the geometric structure that observation requires.
\end{abstract}

\clearpage

% --- SECTION 1 ---
\section{Introduction: A Framework for the Crisis}

Cosmology has reached an impasse. Despite unprecedented observational precision, fundamental measurements refuse to align:
\begin{itemize}
    \item The Hubble constant differs by $\sim 9\%$ between early and late universe measurements \cite{Riess2022}.
    \item Baryon Acoustic Oscillations indicate flatness while CMB power spectra suggest closure \cite{Planck2018}.
    \item Large-angle correlations vanish where theory demands them \cite{Planck2018}.
    \item The quadrupole-octupole alignment defies statistical expectation \cite{Luminet2003}.
\end{itemize}
The field searches for new physics, hidden systematics, or exotic components. We propose something more fundamental: the geometric framework itself is incomplete.

Standard cosmology \cite{Weinberg2008} restricts itself to simply-connected topologies—spaces where every loop contracts to a point. This restriction, never derived from observation, excludes half of all possible compact 3-manifolds \cite{Ellis1971}. The multiply-connected alternatives, where certain loops cannot contract, have been systematically ignored.

This paper introduces \textbf{Projective Reality (PR)}, a complete geometric framework for cosmological structure. This work builds on related geometric foundations, such as the emergence of global curvature from local expansion constraints \cite{Prill2024}. We prove that PR emerges necessarily from observational incidence—how we actually perceive spatial relationships. We then establish its fundamental properties: that its topology must be $\RP^3$ and its invariant must be the VCR. These properties are not assumptions or fits to data; they are mathematical consequences of observation itself.

% --- SECTION 2 ---
\section{Projective Reality: Definition and Derivation of Properties}

\subsection{The Framework of Projective Reality}

\begin{definition}[Projective Reality]
\textbf{Projective Reality (PR)} is the complete geometric framework determined by observational incidence in three-dimensional space, encompassing both the topological manifold and its fundamental invariants.
\end{definition}

PR begins from what we directly observe about spatial structure—not metrics or curvatures \cite{MisnerThorneWheeler1973}, but the incidence relations that define how we perceive space \cite{Coxeter1964}:

\begin{axiom}[Axioms of Visual Incidence]
\label{ax:incidence}
\
\begin{enumerate}
    \item Any two distinct points determine a unique line.
    \item Any three non-collinear points determine a unique plane.
    \item Any two distinct planes meet in a line.
    \item There exist four points, no three collinear and no four coplanar.
\end{enumerate}
\end{axiom}

These are not postulates but empirical facts about spatial perception \cite{VeblenYoung1910}. PR is the unique geometric framework compatible with these observations when embedded in a 3-manifold. We now prove that this framework has two fundamental properties.

\subsection{Property 1: The \texorpdfstring{$\RP^3$}{RP3} Topology}

\begin{theorem}[Topology of PR]
\label{thm:topology}
The manifold of Projective Reality, defined by Axioms \ref{ax:incidence}, is \textbf{Real Projective 3-Space ($\RP^3$)}.
\end{theorem}

\begin{proof}
By the Veblen-Young theorem \cite{VeblenYoung1910}, any incidence structure satisfying Axioms \ref{ax:incidence} with dimension $\geq 3$ is isomorphic to $\PP^n(K)$ for some division ring $K$.

For our three-dimensional case, we have a manifold isomorphic to $\PP^3(K)$. The real dimension of this space is $3d$ where $d = \dim_{\R}(K)$.
Since physical space is three-dimensional:
$$3d = 3 \implies d = 1 \implies K \cong \R$$
Therefore, the manifold is $\PP^3(\R) = \RP^3$.
\end{proof}

\begin{corollary}
PR inherits all topological properties of $\RP^3$ \cite{Munkres2000}:
\begin{itemize}
    \item \textbf{Compact} (finite volume, no boundaries).
    \item \textbf{Multiply-connected} ($\pi_1(\RP^3) = \Z_2$).
    \item \textbf{Locally Euclidean} (indistinguishable from $\R^3$ at small scales).
    \item \textbf{Globally non-orientable}.
\end{itemize}
\end{corollary}

\subsection{Property 2: The VCR Invariant}

In PR, distances and angles are coordinate-dependent \cite{Coxeter1964}. We need an invariant quantity preserved under all projective transformations.

\begin{definition}[Volumetric Cross-Ratio]
For four collinear points $(a,b,c,d)$ and auxiliary points $(e,f)$ not coplanar with the line, the \textbf{Volumetric Cross-Ratio (VCR)} is:
$$\VCR(a,b;c,d) = \frac{[e,f,a,d] \cdot [e,f,b,c]}{[e,f,a,c] \cdot [e,f,b,d]}$$
where $[p,q,r,s]$ denotes the signed 4-volume (determinant) of tetrahedron $pqrs$.
\end{definition}

\begin{theorem}[Invariant of PR]
\label{thm:invariant}
The VCR is the fundamental invariant of Projective Reality, preserved under all projective transformations.
\end{theorem}

\begin{proof}
The VCR is invariant because:
\begin{itemize}
    \item \textbf{Independence from representatives:} Scaling any point's homogeneous representative scales its column in the determinant, applying the same factor to both the numerator and denominator, which cancels.
    \item \textbf{Independence from auxiliary points:} Follows from the cross-ratio reduction in any affine chart.
    \item \textbf{Projective invariance:} Under a transformation $T \in \mathrm{PGL}(4,\R)$, each bracket $[p,q,r,s]$ is multiplied by $\det(T)$. The $\det(T)^2$ factor in the numerator cancels the $\det(T)^2$ factor in the denominator.
\end{itemize}
The VCR is unique (up to functional dependence) by the fundamental theorem of projective invariants \cite{VeblenYoung1910}.
\end{proof}

% --- SECTION 3 ---
\section{How Projective Reality Resolves the Paradoxes}
The apparent contradictions in cosmological data are built into PR's structure.

\subsection{Local Flatness vs Global Closure}
In $\RP^3$, every sufficiently small region is indistinguishable from flat $\R^3$. However, the global topology closes through antipodal identification. This is precisely what observations show:
\begin{itemize}
    \item \textbf{BAO} and other local probes $\implies$ \textbf{flat} \cite{Planck2018}.
    \item \textbf{CMB} and large-scale structure $\implies$ \textbf{closed} \cite{Planck2018}.
\end{itemize}

\subsection{Large-Angle Anomalies}
The $\Z_2$ fundamental group of $\RP^3$ imposes parity selection rules on eigenmodes. Functions on $\RP^3$ must be even on the covering $S^3$, suppressing odd-$\ell$ multipoles and creating the observed alignments \cite{Luminet2003, Weeks2004}.

\subsection{The "Axis of Evil"}
The projective identification structure naturally creates preferred directions—not from anisotropic physics but from the topology itself. The mysterious CMB alignments \cite{Planck2018} are a signature of PR.

% --- SECTION 4 ---
\section{Falsifiable Predictions of Projective Reality}
PR makes five specific, binary predictions derived solely from topology.

\begin{enumerate}
    \item \textbf{Antipodal Matched Circles in the CMB} \\
    \textit{What PR predicts:} The CMB contains pairs of circles with antipodal centers, equal radii, and correlated temperature patterns (matching for even $\ell$, inverted for odd $\ell$) \cite{Lehoucq1996}. \\
    \textit{Why PR requires this:} The last-scattering surface in the covering $S^3$ intersects itself when projected to $\RP^3$. \\
    \textit{How to falsify:} Search with antipodal priors. No statistically significant matches kills PR.

    \item \textbf{Low-$\ell$ Parity Signatures} \\
    \textit{What PR predicts:} Suppression of odd-$\ell$ power relative to even-$\ell$, specific multipole alignments, and broken statistical isotropy. \\
    \textit{Why PR requires this:} Eigenmodes on $\RP^3$ must respect the $\Z_2$ symmetry \cite{Luminet2003}. \\
    \textit{How to falsify:} Full-sky parity analysis showing no asymmetry kills PR.

    \item \textbf{Transient Ghost Images} \\
    \textit{What PR predicts:} Some GRBs/SNe/FRBs appear twice at antipodal positions with identical spectra and predictable delays. \\
    \textit{Why PR requires this:} Light can reach us via both direct and "around-the-universe" paths. \\
    \textit{How to falsify:} Deep surveys finding no antipodal pairs kills PR.

    \item \textbf{Number Count Correlations} \\
    \textit{What PR predicts:} Weak excess correlation between antipodal sky regions in deep galaxy counts \cite{Lehoucq1996}. \\
    \textit{Why PR requires this:} Antipodal regions are the same space viewed from different directions. \\
    \textit{How to falsify:} Uniform counts without antipodal structure kills PR.

    \item \textbf{Polarization Parity} \\
    \textit{What PR predicts:} Nonzero TB and EB correlations where standard cosmology forbids them. \\
    \textit{Why PR requires this:} The projective structure imposes parity constraints on tensor modes. \\
    \textit{How to falsify:} Standard parity behavior in precision polarization maps kills PR.
\end{enumerate}

% --- SECTION 5 ---
\section{Mathematical Completeness}

For completeness, we state the full characterization:
\begin{theorem}[Complete Characterization of PR]
\label{thm:complete}
Projective Reality is the unique geometric framework with:
\begin{enumerate}
    \item Manifold topology $\RP^3$.
    \item Symmetry group $\mathrm{PGL}(4,\R)$.
    \item Fundamental invariant VCR (unique up to functional dependence).
\end{enumerate}
This fully specifies PR as a geometric framework. All physics within PR must respect these structures.
\end{theorem}

% --- SECTION 6 ---
\section{Discussion: The Path Forward}

\subsection{What We Have Established}
We have introduced Projective Reality as a complete geometric framework and proven:
\begin{itemize}
    \item Its topology is necessarily $\RP^3$.
    \item Its invariant is necessarily the VCR.
    \item It naturally explains all cosmological "anomalies."
    \item It makes specific, falsifiable predictions.
\end{itemize}

\subsection{What Remains}
This paper establishes the geometric stage. Future work must:
\begin{itemize}
    \item Reformulate dynamics to preserve VCR rather than metric structures.
    \item Derive field equations respecting projective symmetry.
    \item Understand quantum mechanics within PR.
    \item Reinterpret cosmic "expansion" as a projective phenomenon.
\end{itemize}

\subsection{The Stakes}
If confirmed, PR doesn't just solve paradoxes—it fundamentally reframes physics. The crisis in cosmology becomes the doorway to a deeper geometric understanding of reality.

% --- SECTION 7 ---
\section{Conclusion}

We have introduced Projective Reality, a geometric framework that emerges necessarily from observational incidence. We proved its topology must be $\RP^3$ and its invariant must be the VCR. These are not choices but mathematical consequences of how we perceive space.

The "anomalies" plaguing cosmology are not errors but the signatures of PR. The universe isn't open or closed in the conventional sense \cite{Einstein1917, deSitter1917}—it is projective.

The tests are clear, the predictions are sharp, and the mathematics is rigorous. Within a decade, we will know if Projective Reality is the true geometric framework of the cosmos.

\vspace{1em}
\textit{“The universe is not described by projective geometry—it IS projective geometry. Welcome to Projective Reality.”}

% --- REFERENCES ---
\clearpage
\begin{thebibliography}{99}

\bibitem{Prill2024}
O.~Wojciechowski-Prill,
``Emergence of Gaussian Curvature from Expanding Circle Packings: A Differential Geometric and Projective Perspective,''
\textit{Zenodo}, 2024.
DOI: \href{[https://doi.org/10.5281/zenodo.17439903](https://doi.org/10.5281/zenodo.17439903)}{10.5281/zenodo.17439903}.

\bibitem{VeblenYoung1910}
O.~Veblen and J.~W.~Young,
\textit{Projective Geometry, Vol.~I},
Ginn and Company, 1910.

\bibitem{Coxeter1964}
H.~S.~M.~Coxeter,
\textit{Projective Geometry},
University of Toronto Press, 1964.

\bibitem{Munkres2000}
J.~R.~Munkres,
\textit{Topology}, 2nd ed.,
Prentice Hall, 2000.

\bibitem{MisnerThorneWheeler1973}
C.~W.~Misner, K.~S.~Thorne, and J.~A.~Wheeler,
\textit{Gravitation},
W.~H.~Freeman and Company, 1973.

\bibitem{Weinberg2008}
S.~Weinberg,
\textit{Cosmology},
Oxford University Press, 2008.

\bibitem{Einstein1917}
A.~Einstein,
``Cosmological Considerations in the General Theory of Relativity,''
\textit{Sitzungsberichte der Königlich Preussischen Akademie der Wissenschaften (Berlin)}, 
pp.~142--152, 1917.

\bibitem{deSitter1917}
W.~de~Sitter,
``On the Relativity of Inertia: Remarks Concerning Einstein’s Latest Hypothesis,''
\textit{Proceedings of the Royal Netherlands Academy of Arts and Sciences}, 
vol.~19, pp.~1217--1225, 1917.

\bibitem{Ellis1971}
G.~F.~R.~Ellis,
``Topology and Cosmology,''
\textit{General Relativity and Gravitation}, 
vol.~2, pp.~7--21, 1971.

\bibitem{Lehoucq1996}
R.~Lehoucq, J.~P.~Luminet, and J.~P.~Uzan,
``Limits of Topology Detection in Cosmology,''
\textit{Astronomy and Astrophysics}, 
vol.~313, pp.~339--346, 1996.

\bibitem{Luminet2003}
J.~P.~Luminet, J.~R.~Weeks, A.~Riazuelo, R.~Lehoucq, and J.~P.~Uzan,
``Dodecahedral Space Topology as an Explanation for Weak Wide-Angle Temperature Correlations in the Cosmic Microwave Background,''
\textit{Nature}, 
vol.~425, pp.~593--595, 2003.

\bibitem{Weeks2004}
J.~R.~Weeks,
``The Poincaré Dodecahedral Space and the Mystery of the Missing Fluctuations,''
\textit{Notices of the American Mathematical Society}, 
vol.~51, pp.~610--619, 2004.

\bibitem{Planck2018}
Planck Collaboration,
``Planck 2018 Results VI: Cosmological Parameters,''
\textit{Astronomy \& Astrophysics}, 
vol.~641, A6, 2020.

\bibitem{Riess2022}
A.~G.~Riess \textit{et~al.},
``A Comprehensive Measurement of the Local Value of the Hubble Constant with 1 km s$^{-1}$ Mpc$^{-1}$ Uncertainty from the Hubble Space Telescope and the SH0ES Team,''
\textit{Astrophysical Journal Letters}, 
vol.~934, L7, 2022.

\end{thebibliography}

\end{document}